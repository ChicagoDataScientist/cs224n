\documentclass{article}
\usepackage[margin=1in]{geometry}
\usepackage[utf8]{inputenc}
\usepackage{amsmath} 
\usepackage{natbib}
\usepackage{graphicx}
\usepackage{amsmath,amsfonts,amssymb, hyperref}

\title{Stanford CS 224n Assignment 2}
\author{Hanchung Lee}
\date{January 19, 2020}
\begin{document}

\maketitle
\section{Written: Understanding word2Vec (23 points)}
(a) (3 points) Show that the naive-softmax loss given in Equation (2) is the same as the cross-entropy loss
between $y$ and $\hat{y}$; i.e., show that $$-\sum_{w\in{Vocab}}^{} y_w\log{\hat{y}_w} = -\log{\hat{y}_o}$$ 
\indent Your answer should be one line.
\bigbreak

\noindent
\textbf{Answer:} Since $y$ is a one-hot vector where $y_w = 1$ when $w = o$ and $y_w = 0$ when $w \ne o$, 
\begin{align*}
-\sum_{w\in Vocab}y_w\log(\hat{y}_w) & = -[y_1\log(\hat{y}_1) + \cdots + y_o\log(\hat{y}_o) + \cdots + y_w\log(\hat{y}_w)] \\
& = - y_o\log(\hat{y}_o) \\
& = -\log(\hat{y}_o) \\
\end{align*}
\bigbreak

\noindent (b) (5 points) Compute the partial derivative of $J_{naive-softmax}(v_c,o,U)$ with respect to $v_c$.  Please write your answer in terms of $y$, $\hat{y}$, and $U$.
\bigbreak
\textbf{Answer:} Let input vector be $\theta = U^\intercal v_c$ and prediction function be $\hat{y} = softmax(\theta)$. From (a), we know the derivative of cross-entropy loss is equivalent to softmax loss for one hot vector $\mathbf{y}$, therefore,  

\begin{align*}
J &= CrossEntropy(y, \hat{y})\\
\therefore \frac{\partial J}{\partial \theta} &= (\hat{y} - y)^\intercal\\
\end{align*}
\indent Reference: \url{https://deepnotes.io/softmax-crossentropy}
\smallbreak
From above, we can use chain rule to solve the derivative:

\begin{align*}
    \frac{\partial J}{\partial v_c} &= \frac{\partial J}{\partial \theta} \frac{\partial \theta}{\partial v_c} \\
    &= (\hat{y} - y)^\intercal \frac{\partial U^\intercal v_c}{\partial v_c} \\
    &= U (\hat{y} - y)
\end{align*}

\smallbreak

\bigbreak

\noindent(c)  (5 points) Compute the partial derivatives of $J_{naive-softmax}(\mathbf{v_c},o,\mathbf{U})$ with respect to each of the ‘outside’ word vectors, $u_w$’s.  There will be two cases:  when $w=o$, the true ‘outside’ word vector, and $w\ne{o}$, for all other words.  Please write you answer in terms of $y$, $\hat{y}$, and $v_c$.
\bigbreak

\textbf{Answer:} Similar to answer (b) above.
\begin{align*}
    \frac{\partial J}{\partial v_c} &= \frac{\partial J}{\partial \theta} \frac{\partial \theta}{\partial v_c}\\
    &= (\hat{y} - y) \frac{\partial \mathbf{U^\intercal}v_c}{\partial \mathbf{U}} \\
    &= (\hat{y} - y) v_c
\end{align*}
\bigbreak


\noindent(d)  (3 Points) The sigmoid function is given by Equation 4:
\begin{equation} \label{eq4}
\sigma(x) = \frac{1}{1+e^{-x}} =  \frac{e^{x}}{e^{x} + 1}
\end{equation}
Please compute the derivative of $\sigma(x)$ with respect to $x$, where $x$ is a scalar.  Hint:  you may want to write your answer in terms of $\sigma(x)$.
\bigbreak
\textbf{Answer:} Apply quotient rule $f(x) = \frac{g(x)}{h(x)}$, $f\prime(x) = \frac{g\prime(x)h(x) + g(x)h\prime(x)}{h^2(x)}$, we have:
\begin{align*}
\frac{\partial \sigma(x)}{\partial x} &= \frac{\partial \frac{e^x}{e^x+1}}{\partial x}\\
&=\frac{(e^x+1)e^x - e^x e^x}{(e^x+1)^2}\\
&=\frac{e^x}{(e^x + 1)^2}\\
&=\sigma(x) \frac{1}{e^x+1}\\
&=\sigma(x) \frac{-e^x + e^x + 1}{e^x+1}\\
&=\sigma(x) (\frac{-e^x}{e^x+1} + \frac{e^x + 1}{e^x+1} )\\
&=\sigma(x) (1 - \sigma(x))
\end{align*}
\bigbreak

\noindent(e)  (4 points) Now we shall consider the Negative Sampling loss, which is an alternative to the Naive Softmax loss. Assume that $K$ negative samples (words) are drawn from the vocabulary. For simplicity of notation we shall refer to them as $w_1, w_2, \cdots,w_K$ and their outside vectors as $\mathbf{u_1},\cdots,\mathbf{u_K}$. Note that $o \not\in {w_1,...,w_K}$.  For a center word $c$ and an outside word $o$, the negative sampling loss function is given by:
$$J_{neg-sample}(v_c, o, U) = -\log (\sigma (u_o^\intercal v_c)) - \sum_{k=1}^{K} \log \sigma(-u_k^\intercal v_c))$$
for a sample $w_1,\cdots,w_K$, where $\sigma(.)$ is the sigmoid function.
\smallbreak
Please repeat parts (b) and (c), computing the partial derivatives of $J_{neg-sample}$ with respect to $v_c$, with respect to $u_o$, and with respect  to a negative sample $u_k$. Please write your answers in terms of the vectors $u_o$, $v_c$, and $u_k$, where $k \in [1, K]$. After you’ve done this, describe with one sentence why this loss function is much more efficient to compute than the naive-softmax loss. Note, you should be able to use your solution to part (d) to help compute the necessary gradients here.
\bigbreak
\noindent
\textbf{Answer:}
\begin{align*}
\frac{\partial J}{\partial u_o} &= (\sigma(u_o^\intercal v_c) - 1)v_c \\
\frac{\partial J}{\partial u_k} &= (\sigma(u_k^\intercal v_c) - 1)v_c, \forall k \in [1, K] \\
\frac{\partial J}{\partial v_c} &=  (\sigma(u_o^\intercal v_c) - 1)u_o - \sum_{k=1}^{K}(\sigma(-u_k^\intercal v_c)-1)u_k
\end{align*}
\textbf{Answer:} For naive softmax loss, it computes the whole outside vectors $U$.  But for negative sampling loss, it only calculates a fixed size K. Thus, negative sampling loss is more compute and memory efficient
\bigbreak

\noindent(f)  (3 points) Suppose the center word $w_t$ and the context window is $[w_{t-m}, \cdots, w_{t-1}, w_t, w_{t+1}, \cdots, w_{t+m}]$, where m is the context window size. Recall that for the skip-gram version of word2Vec, the total loss for the context window is:

\begin{equation}
J_{skip-gram}(v_c, w_{t-m}, \cdots, w_{t+m}, U) = \sum_{\substack{-m \leq j \leq m \\ j\ne 0}} J(v_c, w_{t+j}, U)
\end{equation}
Here, $J(v_c, w_{t+j}, U)$ represents an arbitrary loss term for the center word $c = w_t$ and outside word $w_{t+j}$. $J(v_c, w_{t+j}, U)$ could be $J_{naive-softmax}(v_c, w_{t+j}, U)$ or $J{neg-sample}(v_c, w_{t+j}, U)$, depending on your implementation.

\medbreak
Write down three partial derivatives:
\medbreak
(i) $\partial J_{skip-gram}(v_c, w_{t-m}, \cdots, w_{t+m})/\partial U$

(ii) $\partial J_{skip-gram}(v_c, w_{t-m}, \cdots, w_{t+m})/\partial v_c$

(iii) $\partial J_{skip-gram}(v_c, w_{t-m}, \cdots, w_{t+m})/\partial w_c$ when $w \ne c$
\medbreak
Write your answers in terms of $\partial J(v_c, w_{t+j}, U)/\partial U$ and $\partial J(v_c, w_{t+j}, U)/\partial v_c$. This is very simple $-$ each solution should be one line.
\bigbreak
\noindent
\textbf{Answer:} 
\begin{align*}
    \partial J_{skip-gram}(v_c, w_{t-m}, \cdots, w_{t+m})/\partial U &= \sum_{\substack{-m \leq j \leq m\\ j\ne 0}} \frac{J(v_c, w_{t+j}, U)}{\partial U}\\
    \partial J_{skip-gram}(v_c, w_{t-m}, \cdots, w_{t+m})/\partial v_c &= \sum_{\substack{-m \leq j \leq m\\ j\ne 0}} \frac{J(v_c, w_{t+j}, U)}{\partial v_c}\\
    \partial J_{skip-gram}(v_c, w_{t-m}, \cdots, w_{t+m})/\partial w_c (w \ne c) &= 0
\end{align*}

\bigbreak
\end{document}